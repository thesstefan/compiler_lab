\documentclass{article}[12pt]
\usepackage[cache=false]{minted}
\usepackage[margin=1in]{geometry}
\usepackage{tcolorbox}

\title{\textbf{UniC Language Examples}}
\author{Stefan Stefanache (936/2)}
\date{}

\begin{document}
    \maketitle

    \textbf{UniC} provides a small subset of the C language, with a stricter
    syntax and some new helper functions such as \mintinline{c}{read} and \mintinline{c}{write}.

    \begin{tcolorbox}[
        standard jigsaw,
        title=P1. Compute the maximum of 3 numbers,
        opacityback=0]
    \begin{minted}{c}
        int a; read(a);
        int b; read(b);
        int c; read(c);

        int max = a;

        if (max < b) {
            max = b;
        }

        if (max < c) {
            max = c;
        }

        write("Max is: ");
        write(max);
    \end{minted}
    \end{tcolorbox}

    \begin{tcolorbox}[
        standard jigsaw,
        title=\textbf{ERR\_P1. Compute the maximum of 3 numbers (with 2 errors)},
        colframe=red,
        coltitle=black,
        opacityback=0]
    \begin{minted}{c}
        // Error 1: read puts input in the given variable
        int a = read();
        int b; read(b);
        int c; read(c);

        // Error 2: max is not initialized yet
        int max = max + a; 

        if (max < b) {
            max = b;
        }

        if (max < c) {
            max = c;
        }

        write("Max is: ");
        write(max);
    \end{minted}
    \end{tcolorbox}

    \begin{tcolorbox}[
        standard jigsaw,
        title=P2. Check if a number is prime,
        opacityback=0]
    \begin{minted}{c}
        int n;
        read(n);

        string prime_message = "Number is prime!"
        string not_prime_message = "Number is NOT prime!"

        if (n == 2 || n == 3) {
            write(prime_message);

            return;
        }

        if (n <= 1 || n % 2 == 0 || n % 3 == 0) {
            write(not_prime_message);
        
            return;
        } 
        
        for (int i = 5; i * i <= n; i += 6) {
            if (n % i == 0 || n % (i + 2) == 0) {
                write(not_prime_message);

                return;
            }
        }

        write(prime_message);
    \end{minted}
    \end{tcolorbox}

    \begin{tcolorbox}[
        standard jigsaw,
        title=P3. Compute the sum of $n$ elements,
        opacityback=0]
    \begin{minted}{c}
        int n;
        read(n);

        int sum = 0;
        int x;
        for (int i = 0; i < n; i++) {
            read(x);
            sum = sum + x;
        }

        write(sum);
    \end{minted}
    \end{tcolorbox}

    \vspace{2em}
\end{document}
